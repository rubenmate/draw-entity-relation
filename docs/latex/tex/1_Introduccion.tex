\capitulo{1}{Introducción}

En desarrollo de software el correcto modelado de la información con la que se trata es fundamental. Una de las técnicas más utilizadas para representar la estructura de un modelo de datos, previo a su implementación en base de datos, es el modelo de diagramas entidad-relación (E-R) \cite{journals/tods/Chen76}. Estos diagramas permiten visualizar las relaciones entre diferentes entidades, así como sus atributos, facilitando así la comprensión y diseño de sistemas de información complejos.

En la era actual, muy centrada en el entorno web surge la necesidad de tener una aplicación web que nos permita, de manera sencilla y accesible en cualquier dispositivo, modelar diagramas entidad-relación y exportarlos fácilmente a tablas SQL. Esta herramienta tiene también un propósito académico, sirve para aprender respecto al paso a tablas SQL de un diagrama E-R.

El objetivo de este proyecto es desarrollar una aplicación web que permita modelar diagramas entidad-relación de manera sencilla, utilizando la biblioteca mxGraph en un entorno React. La aplicación no solo permite la creación y edición de estos diagramas, sino que también permite validarlos y exportarlos a un script SQL listo para ser ejecutado.

Las características principales de esta versión de la herramienta incluyen:
\begin{itemize}
\tightlist
    \item Aplicación web single page application (SPA) usando el framework React.
    \item Interfaz de usuario simple y accesible que permite a los usuarios diseñar diagramas E-R de manera visual.
    \item Uso de la biblioteca mxGraph para el modelado de los diagramas.
    \item Soporte para modelado y relaciones más básicas de E-R como son:
    \begin{itemize}
    \tightlist
        \item Interrelaciones binarias.
        \item Interrelaciones reflexivas.
    \end{itemize}
    \item Validación de los diagramas para asegurar la consistencia y corrección de los modelos.
    \item Exportación e importación de los diagramas validados a un archivo JSON, permitiendo su recuperación para posteriores ediciones.
    \item Exportación de los diagramas validados a scripts SQL, facilitando así su implementación en bases de datos reales.
\end{itemize}

La creación de esta aplicación responde a la necesidad de modernizar y simplificar el proceso de modelado de bases de datos, haciendo que sea accesible para un público más amplio y sin requerir de aplicaciones específicas bastando para ello tan solo un navegador web.

En este documento se detalla el desarrollo de la aplicación, desde su concepción y diseño hasta su implementación y validación.

\section{Estructura de la memoria}
\begin{itemize}
\tightlist
    \item
        \textbf{Introducción: }Descripción de la situación y el tema sobre el que va a tratar el proyecto. Estructura de la memoria y de los anexos.
    \item 
        \textbf{Objetivos del proyecto: }Explicación de los temas que se persiguen en este proyecto.
    \item 
        \textbf{Conceptos teóricos: }Introducción a determinados conceptos necesarios para la comprensión del proyecto.
    \item 
        \textbf{Técnicas y herramientas: }Presentación de metodologías y herramientas que han sido utilizadas para llevar a cabo el proyecto. 
    \item 
        \textbf{Aspectos relevantes del desarrollo del proyecto: }Aspectos a destacar durante el desarrollo del proyecto
    \item 
        \textbf{Trabajos relacionados: } Pequeña introducción a proyectos similares y comparativa.
    \item 
        \textbf{Conclusiones y líneas de trabajo futuras: }Conclusiones obtenidas al final del proyecto y posibles ideas futuras.
\end{itemize}

\section{Estructura de los anexos}
\begin{itemize}
\tightlist
    \item 
        \textbf{Plan de proyecto software: }Planificación temporal y viabilidad económica y legal.
    \item 
        \textbf{Especificación de requisitos: }Objetivos y requisitos a completar durante el desarrollo del proyecto.
    \item 
        \textbf{Especificación de diseño: }Recoge los diseños de datos, procedimental, arquitectónico y de interfaces.
    \item 
        \textbf{Documentación técnica de programación: }Guía técnica del proyecto con conceptos como instalación, organización de carpetas, ejecución del proyecto y pruebas.
    \item 
        \textbf{Documentación de usuario: }Gúia destinada al usuario final con explicación paso a paso referente al uso de la aplicación.
    \item 
        \textbf{Anexo de sostenibilización curricular: }Reflexión personal sobre los aspectos de sostenibilidad que se abordan en el trabajo.
\end{itemize}

\section{Enlaces}
\begin{itemize}
\tightlist
    \item 
        Página web del repositorio en GitHub \cite{draw-er-app:repo}
    \item 
        Despliegue de la aplicación en Vercel \cite{draw-er-app:web}
\end{itemize}

