\apendice{Anexo de sostenibilización curricular}

\section{Introducción}
Este anexo incluye una reflexión personal sobre los aspectos de sostenibilidad abordados en este Trabajo de Fin de Grado (TFG). Se puede considerar que la sostenibilidad es una idea crucial en la formación académica actual y que además es fundamental para la creación de soluciones tecnológicas que respeten y promuevan el bienestar ambiental, social y económico.

A continuación, se detallan algunas de las competencias de sostenibilidad adquiridas y aplicadas durante el desarrollo del TFG, relacionándolas con los Objetivos de Desarrollo Sostenible (ODS) de la ONU.

\section{Competencias de sostenibilidad adquiridas}

\subsection{ODS 4: Educación de Calidad}
Durante el desarrollo de este TFG, se han adquirido  y aplicado conocimientos avanzados en el ámbito del modelado de diagramas E-R y el desarrollo de aplicaciones web. Un desarrollo implementando buenas prácticas, tales como el uso de metodologías ágiles y herramientas de integración y despliegue continuos, contribuyen a una educación de calidad. Estos nuevos conocimientos no mejora simplemente mis habilidades y conocimientos técnicos, sino que también estimulan un enfoque crítico y reflexivo al abordar la resolución de problemas complejos.

\subsection{ODS 9: Industria, Innovación e Infraestructura}
Este proyecto que se acaba de tratar se centra en la creación de una aplicación web para el modelado de diagramas E-R. Este tipo de herramientas son importantes para la educación, digitalización y modernización de infraestructuras educativas y empresariales. Al facilitar el aprendizaje, diseño y validación de bases de datos; esta aplicación promueve la eficiencia en esta tarea a la vez que se contribuye a la construcción de infraestructuras digitales resilientes y sostenibles.

\subsection{ODS 12: Producción y Consumo Responsables}
El desarrollo de este TFG ha incluido la utilización de recursos de manera eficiente y responsable. Se han usado librerías ya desarrolladas, como son React y mxGraph. Esto reduce la necesidad de recursos adicionales y promueve la reutilización de componentes existentes.

\subsection{ODS 13: Acción por el Clima}
Aunque no se puede decir que este TFG esté directamente relacionado con la mitigación del cambio climático, sí se puede decir que la adopción de prácticas sostenibles en el desarrollo de software tiene un impacto positivo en el medio ambiente.
El uso de herramientas de integración continua y despliegue continuo reduce la necesidad de infraestructuras físicas y, como consecuencia, disminuye la huella de carbono asociada al desarrollo del software.

\subsection{ODS 17: Alianzas para Lograr los Objetivos}
El desarrollo del TFG ha implicado la colaboración con varias partes interesadas, como por ejemplo tutores académicos y otros desarrolladores con problemáticas comunes. Esta colaboración ha fomentado el intercambio de conocimientos, contribuyendo a la construcción de alianzas. El enfoque colaborativo y la comunicación son competencias clave que se han desarrollado y aplicado durante el proyecto, y que también son clave para el desarrollo sostenible.

\section{Reflexión final}
Como conclusión, el desarrollo de este TFG ha permitido la adquisición y aplicación de ciertas competencias de sostenibilidad que están alineadas con varios ODS (Objetivos de Desarrollo Sostenible). 
La implementación de buenas prácticas de desarrollo, la promoción de la innovación y la eficiencia, y el enfoque en la sostenibilidad ambiental y social son aspectos que han tenido presencia en el proyecto de una u otra manera. 
Esta experiencia ha mejorado mis habilidades técnicas. Pero no solo eso, también ha fortalecido mi compromiso con el desarrollo sostenible y responsable en el ámbito tecnológico.

Para más información sobre la sostenibilización curricular, se puede consultar el documento de la CRUE \url{https://www.crue.org/wp-content/uploads/2020/02/Directrices_Sosteniblidad_Crue2012.pdf}.
