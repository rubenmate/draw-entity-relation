\apendice{Plan de Proyecto Software}

\section{Introducción}
En este apartado se realiza un pequeño resumen del desarrollo del proyecto y de los diferentes sprints que lo han compuesto, así como un estudio de viabilidad en el que se presenta tanto la viabilidad económica como legal del proyecto.

\section{Planificación temporal}
Se ha usado una metodología de desarrollo ágil, los sprints han sido de aproxidamente 3-4 semanas para poder compaginarlo con mi trabajo.
Los sprints se han administrado con las Milestones de la plataforma GitHub en combinación con las labels que determinan el tipo de tarea que es cada issue.
\imagen{sprints}{División en sprints del desarrollo}
\imagen{labels}{Labels a asignar a los diferentes issues y pull requests}

\subsection{Sprint 0 - 13/03/2024-8/04/2024 }
\imagen{sprint-0}{Issues del Sprint 0}{1}
\begin{table}[ht!]
    \centering
    \resizebox{14cm}{!} {
    \begin{tabular}{|l|c|}
    \hline
    \rowcolor[rgb]{0.9, 0.9, 0.9}
    \textbf{Issues completadas} & \textbf{Label} \\ \hline
    \textbf{Decision: Canvas Drawing Library Selection} & \cellcolor[rgb]{0.83, 0.77, 0.97}research \\ \hline 
    \textbf{Consideration: Framework Selection - React vs Vanilla Javascript} & \cellcolor[rgb]{0.83, 0.77, 0.97}research \\ \hline 
    \end{tabular}}
    \caption{Issues completadas Sprint 0}
    \label{tab:my_label}
\end{table}




\subsection{Sprint 1}
\imagen{sprint-1}{Issues del Sprint 1}{1}

\subsection{Sprint 2}
\imagen{sprint-2}{Issues del Sprint 2}{1}

\subsection{Sprint 3}
\imagen{sprint-3}{Issues del Sprint 3}{1}

\subsection{Sprint 4}
\imagen{sprint-4}{Issues del Sprint 4}{1}

\section{Estudio de viabilidad}

\subsection{Viabilidad económica}

\subsection{Viabilidad legal}
Licencias de los proyectos usados
Material UI
mxGraph
React

Hacer tabla comparativa con las licencias de los proyectos en busca de si hay alguna que no permita licencia comercial.


