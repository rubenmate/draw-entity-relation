\apendice{Documentación de usuario}

\section{Introducción}
En este apartado se proporciona una guía completa para los usuarios de la aplicación. Se detallan los requisitos necesarios, el proceso de instalación y un manual de usuario para facilitar el uso de la aplicación.

\section{Requisitos de usuarios}
Para utilizar esta aplicación, los usuarios deben cumplir con los siguientes requisitos:
\begin{itemize}
    \item Disponer de un navegador web actualizado (Chrome, Firefox, Safari, Edge o similar).
    \item Conexión a internet.
    \item Es conveniente tener algún conocimiento del modelo E-R.
\end{itemize}

\section{Instalación}
La aplicación se puede utilizar directamente desde el navegador sin necesidad de instalación y es el modo recomendado de uso \footnote{\url{https://draw-entity-relation.vercel.app}}. Sin embargo, para aquellos usuarios que deseen ejecutar la aplicación localmente, se pueden seguir estos pasos:

\begin{enumerate}
    \item Clonar el repositorio del proyecto desde GitHub:
    \begin{verbatim}
    git clone https://github.com/rubenmate/draw-entity-relation.git
    \end{verbatim}
    \item Acceder al directorio del proyecto:
    \begin{verbatim}
    cd draw-entity-relation
    \end{verbatim}
    \item Instalar las dependencias necesarias:
    \begin{verbatim}
    npm install
    \end{verbatim}
    \item Iniciar el servidor de desarrollo:
    \begin{verbatim}
    npm start
    \end{verbatim}
    \item Abrir navegador web y acceder a \texttt{http://localhost:3000}.
\end{enumerate}

Para levantar la aplicación en un entorno de producción, se puede utilizar la herramienta \texttt{serve} de npm \cite{npm:serve}:
\begin{enumerate}
    \item Instalar \texttt{serve} a nivel global en el sistema:
    \begin{verbatim}
    npm install -g serve
    \end{verbatim}
    \item Hacer build del proyecto:
    \begin{verbatim}
    npm run build
    \end{verbatim}
    \item Servir la aplicación con \texttt{serve}:
    \begin{verbatim}
    serve -s build
    \end{verbatim}
\end{enumerate}

\section{Manual del usuario}
Este manual proporciona una guía paso a paso sobre cómo utilizar las principales funcionalidades de la aplicación.

\subsection{Inicio}
Al entrar en la aplicación encontraremos las diferentes partes de la aplicación y el canvas vacío.
Con el click derecho del ratón se puede hacer drag del canvas y movernos por si necesitásemos más espacio.

\imagen{app}{Pantalla inicial de la aplicación.}

\imagen{app-parts}{Diferentes componentes de la aplicación.}

\subsection{Crear un nuevo diagrama E-R}
\begin{enumerate}
    \item En caso de tuviéramos un diagrama en proceso de modelar el proceso sería hacer click en el botón \texttt{Reiniciar} y \texttt{Aceptar} en el diálogo.
    \item Con el canvas vacío el proceso de funcionamiento es tan simple como arrastrar los diferentes objetos (Entidades-rectángulos, Relaciones-Rombos) al canvas.
\end{enumerate}

\subsection{Acciones comunes sobre objetos}
Con un objeto cualquiera (Entidades, Atributos y Relaciones) podemos realizar estas acciones:
\begin{enumerate}
    \item Seleccionar un elemento: Al hacer click  sobre un elemento se selecciona.
    \item Desseleccionar un elemento: Al hacer click sobre un área vacía se desselecciona el elemento seleccionado.
    \item Doble click para cambiar de nombre: Al hacer doble click sobre un objeto nos deja usar el teclado para cambiarlo de nombre.
    \item Move back y Move front: Son comunes a todos los objetos y sirven para aquellos casos que queramos recuperar algún objeto que ha quedado detrás de otro. Con \texttt{Move back} el objeto queda por detrás del resto, \texttt{Move front} hace lo contrario.
\end{enumerate}

\subsection{Acciones sobre una Entidad}
Con una entidad seleccionada podemos realizar las siguientes acciones:
\begin{enumerate}
    \item Añadir atributo: Añade un atributo a esta entidad, el primer atributo siempre será clave (no así los siguientes) y la posición de los siguientes siempre será un poco debajo del último atributo añadido.
    \item Ocultar atributos: Oculta los atributos de una entidad para evitar sobrecargar la visualización.
    \item Mostrar atributos: Muestra los atributos de una entidad previamente ocultos.
    \item Borrar: Elimina la entidad así como sus atributos y vinculación con relaciones si los hubiera.
\end{enumerate}

\imagen{entity-actions-app}{Acciones contextuales sobre una entidad.}

\subsection{Acciones sobre un Atributo}
Con un atributo seleccionado (no clave, el atributo clave está protegido) podemos realizar las siguientes acciones:
\begin{enumerate}
    \item Convertir en clave: Convierte el atributo en clave. Como en esta versión no se permiten claves compuestas, el resultado será que la anterior clave pasará a ser atributo normal.
    \item Borrar: Elimina el atributo de la entidad.
\end{enumerate}

\imagen{attribute-actions-app}{Acciones contextuales sobre un atributo.}

\subsection{Acciones sobre una Relación}
Con una relación seleccionada podemos realizar las siguientes acciones:
\begin{enumerate}
    \item Convertir en clave: Convierte el atributo en clave. Como en esta versión no se permiten claves compuestas, el resultado será que la anterior clave pasará a ser atributo normal.
    \item Borrar: Elimina el atributo de la entidad.
\end{enumerate}

\imagen{relation-actions-app}{Acciones contextuales sobre una relación.}

Las relaciones N:M pueden contener atributos y por tanto también muestran los botones \texttt{Añadir atributo} o \texttt{Ocultar atributos} que se comportan del mismo modo que las entidades con la salvedad de que estos atributos no serán clave.

\imagen{n-m-relation-actions-app}{Añadir atributos en una relación N-M.}

\subsubsection{Configurar relaciones}
Para configurar una relación hay que hacer click sobre el botón \texttt{Configurar relación}, se mostrarán dos diálogos con selectores (que muestran las diferentes entidades del diagrama).
Al escoger entidades en los dos lados (pueden ser la misma, se permiten relaciones reflexivas).
Las relaciones se pueden reconfigurar con este mismo procedimiento.

\imagen{configure-relation-app}{Diálogo para configurar relaciones.}

\subsubsection{Configurar cardinalidades}
Para configurar las cardinalidades de los lados de una relación hay que hacer click sobre el botón \texttt{Configurar cardinalidades}. Se mostrarán dos diálogos con selectores (que muestran las posibles cardinalidades).
El caso 1:1-1:1 no está permitido, al escoger 1:1 en uno de los dos lados desaparecerá del otro.
Las cardinalidades se pueden reconfigurar con este mismo procedimiento.

\imagen{configure-cardinalities-app}{Diálogo para configurar cardinalidades de una relación.}

\subsection{Generar script SQL}
\begin{enumerate}
    \item Al hacer click al botón \texttt{Generar SQL} se valida el diagrama internamente.
    \item En caso de que el diagrama sea válido se generará un archivo SQL que se descargará automáticamente.
    \item En caso de no ser válido se mostrará una serie de diagnósticos que están causando que el diagrama no sea válido.
\end{enumerate}

\imagen{validation-sql-app}{Diagnóstico al generar SQL con un diagrama no válido.}

\subsection{Exportar un diagrama a JSON}
El funcionamiento es exactamente igual que para generar un script SQL, pero con la salvedad de que se exporta el diagrama en formato JSON. Este archivo está listo para ser importado y recreado.

\subsection{Importar un diagrama desde JSON}
\begin{enumerate}
    \item Haga clic en el botón \texttt{Importar JSON} y seleccione el archivo JSON que contiene el diagrama.
    \item La aplicación comprobará la validez del diagrama y lo recreará.
\end{enumerate}

\imagen{import-json-app}{Diálogo con selección de archivo para import diagrama JSON.}

\subsection{Reiniciar canvas}
\begin{enumerate}
    \item Haga clic en el botón \texttt{Reiniciar}.
    \item Aparecerá un diálogo de confirmación con las opciones \texttt{Aceptar} y \texttt{Cancelar}.
    \item Al hacer click en \texttt{Aceptar} se borrará el estado del diagrama y se reiniciará el canvas.
\end{enumerate}

\imagen{reset-canvas-app}{Diálogo de confimación para reiniciar el canvas.}

Con estas instrucciones, los usuarios podrán utilizar la aplicación de manera efectiva para crear, validar, exportar e importar diagramas E-R, así como generar scripts SQL a partir de sus diagramas modelados.
